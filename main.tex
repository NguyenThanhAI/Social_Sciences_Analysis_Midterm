\documentclass[14pt, a4paper]{article}
\usepackage{minitoc}
\usepackage[left=3.00cm, right=2.5cm, top=2.00cm, bottom=2.00cm]{geometry}
\usepackage{amsmath}
\usepackage{amssymb}
\usepackage{amsthm}
\usepackage{mathtools}
\usepackage{graphicx}
%\usepackage{algpseudocode}
%\usepackage{algorithm}
\usepackage[ruled,vlined,linesnumbered]{algorithm2e}
\usepackage{blindtext}
\usepackage{setspace}
\usepackage[utf8]{inputenc}
\usepackage[utf8]{vietnam}
\usepackage[center]{caption}
\usepackage[shortlabels]{enumitem}
\usepackage{fancyhdr} % header, footer
\usepackage{hyperref} % loại bỏ border với mục lục và công thức
\usepackage[nonumberlist, nopostdot, nogroupskip]{glossaries}
\usepackage{glossary-superragged}
\usepackage{tikz,tkz-tab}
\usepackage{pythonhighlight}
\setglossarystyle{superraggedheaderborder}
\pagestyle{fancy}
%\usepackage[style=numeric,sortcites]{biblatex}
%\addbibresource{ref.bib}
%\usepackage[numbers]{natbib}
\usepackage{indentfirst}
\usepackage[natbib,backend=biber,style=ieee, sorting=ynt]{biblatex}

\usepackage{caption}
\usepackage{subcaption}

\bibliography{ref.bib}

\graphicspath{{./figures/}}

\fancyhf{}
%\rhead{\textbf{Môn học: Các phương pháp thống kê hiện đại trong nghiên cứu Xã hội học}}
\lhead{\textbf{GVHD: TS. Trịnh Quốc Anh}}
\rfoot{\thepage}
\lfoot{\textbf{Học viên thực hiện: Nguyễn Chí Thanh - 21007925}}
\renewcommand{\headrulewidth}{0.4pt}
\renewcommand{\footrulewidth}{0.4pt}
%
%\numberwithin{equation}{section}
%\numberwithin{algorithm}{section}
%\numberwithin{figure}{section}
%
%\setlength{\parindent}{0.5cm}
%
%\setcounter{secnumdepth}{3} % Cho phép subsubsection trong report
%\setcounter{tocdepth}{3} % Chèn subsubsection vào bảng mục lục

%\newtheorem{dl}{Định lý}
%\newtheorem{md}{Mệnh đề}
%\newtheorem{bd}{Bổ đề}
%\newtheorem{dn}{Định nghĩa}
%\newtheorem{hq}{Hệ quả}

%\newtheorem{baitap}{Bài tập}
%\newtheorem*{loigiai}{Lời giải}

%\numberwithin{dl}{section}
%\numberwithin{md}{section}
%\numberwithin{bd}{section}
%\numberwithin{dn}{section}
%\numberwithin{hq}{section}

\setlength{\parindent}{0cm}

\newtheorem{dl}{Định lý}
\newtheoremstyle{sltheorem}
{}                % Space above
{}                % Space below
{\normalfont}        % Theorem body font % (default is "\upshape")
{}                % Indent amount
{\bfseries}       % Theorem head font % (default is \mdseries)
{.}               % Punctuation after theorem head % default: no punctuation
{ }               % Space after theorem head
{}                % Theorem head spec
\theoremstyle{sltheorem}
\newtheorem{baitap}{Bài tập}
\newtheoremstyle{soltheorem}
{}                % Space above
{}                % Space below
{\normalfont}        % Theorem body font % (default is "\upshape")
{}                % Indent amount
{\bfseries}       % Theorem head font % (default is \mdseries)
{.}               % Punctuation after theorem head % default: no punctuation
{\newline}               % Space after theorem head
{}                % Theorem head spec
\theoremstyle{soltheorem}
\newtheorem*{loigiai}{Lời giải}

\onehalfspacing


\begin{document}
\begin{titlepage}

    \newcommand{\HRule}{\rule{\linewidth}{0.5mm}} % Defines a new command for the horizontal lines, change thickness here

    \center % Center everything on the page

    %----------------------------------------------------------------------------------------
    %	HEADING SECTIONS
    %----------------------------------------------------------------------------------------
    \textsc{\LARGE Đại học Quốc Gia Hà Nội}\\[0.5cm]
    \textsc{\LARGE Trường đại học Khoa học tự nhiên}\\[0.5cm] % Name of your university/college
    \textsc{\LARGE Khoa Toán - Cơ - Tin học}\\[0.5cm]

    \includegraphics[scale=0.2]{HUS-logo.jpg}\\[0.5cm]

    \textsc{\Large Chuyên ngành: Khoa học dữ liệu}\\[0.5cm] % Major heading such as course name


    %----------------------------------------------------------------------------------------
    %	TITLE SECTION
    %----------------------------------------------------------------------------------------

    \HRule \\[0.4cm]
    { \huge \bfseries Bài tập môn học}\\[0.4cm] % Title of your document
    \HRule \\[1.5cm]

    \textsc{\Large Môn học: Các phương pháp thống kê hiện đại \\ trong nghiên cứu Xã hội học}\\[1cm] % Minor heading such as course title


    \textsc{\Large Bài tập giữa kỳ}\\[1cm]


    %----------------------------------------------------------------------------------------
    %	AUTHOR SECTION
    %----------------------------------------------------------------------------------------
    \begin{minipage}{0.4\textwidth}
        \begin{flushleft} \large
        \emph{Giảng viên hướng dẫn:} \\
        TS. Trịnh Quốc Anh % Supervisor's Name
        \end{flushleft}
    \end{minipage}\\[0.5cm]

    \begin{minipage}{0.4\textwidth}
    \begin{flushleft} \large
    \emph{Học viên thực hiện:}\\
    Nguyễn Chí Thanh \\
    MSHV: 21007925 \\ % Your name
    Lớp: Khoa học dữ liệu - K4
    \end{flushleft}
    \end{minipage}


    % If you don't want a supervisor, uncomment the two lines below and remove the section above
    %\Large \emph{Author:}\\
    %John \textsc{Smith}\\[3cm] % Your name

    %----------------------------------------------------------------------------------------
    %	DATE SECTION
    %----------------------------------------------------------------------------------------

    % I don't want day because it is English
    % {\large \today}\\[2cm] % Date, change the \today to a set date if you want to be precise

    %----------------------------------------------------------------------------------------
    %	LOGO SECTION
    %----------------------------------------------------------------------------------------

    %\includegraphics{logo/rsz_3logo-khtn.png}\\[1cm] % Include a department/university logo - this will require the graphicx package

    %----------------------------------------------------------------------------------------

    \vfill % Fill the rest of the page with whitespace

\end{titlepage}

\nocite{*}

\newpage

\begin{baitap}
    Ước lượng Bayes cho tham số của phân phối Poisson.

    Đính kèm là dữ liệu tổng điều tra xã hội (GSS) của Mỹ từ năm 1972 đến năm 1998.

    Người ta quan tâm đến số con trung bình (tỷ lệ sinh) của phụ nữ Mỹ ở độ tuổi 40 trong thập niên 1990 (YEAR$\geq$1990 \& AGE==40 \& FEMALE==1). 
    Có 02 nhóm phụ nữ được nghiên cứu là nhóm có trình độ văn hoá trên phổ thông (DEG$\geq$3) và nhóm còn lại. 
    Biết rằng số con sinh ra tuân theo phân phối Poisson.
    
    Từ dữ liệu và sử dụng thống kê Bayes, hãy:

    \begin{enumerate}
        \item Xác định phân phối của dữ liệu (likelihood)
        \item Xác định phân phối hậu nghiệm của tỷ lệ sinh khi phân phối tiên nghiệm là:
        \begin{enumerate}[label=(\alph*)]
            \item Phân phối đều
            \item Phân phối Gamma
        \end{enumerate}
        \item Nêu đặc điểm của các phân phối hậu nghiệm tìm được
        \item Ước lượng tỷ lệ sinh của 02 nhóm phụ nữ
        \item So sánh tỷ lệ sinh của 2 nhóm và đánh giá sự khác biệt (nếu có)
    \end{enumerate}
\end{baitap}

\begin{loigiai}
    \begin{enumerate}
        \item Xác định phân phối của dữ liệu (likelihood):
        
        Ta biết số lượng con sinh ra tuân theo phân phối Poisson.
        Ta gọi $X_i$ là số con sinh ra của một người phụ nữ thứ $i$ trong tập dữ liệu.
        Ta gọi tỷ lệ sinh là $\lambda$.
        Do $X \sim P(\lambda)$ nên:

        \begin{equation*}
            f(X=k \vert \lambda) = e^{-\lambda} \dfrac{\lambda^k}{k!}
        \end{equation*}

        Ta giả định các quan sát (những người phụ nữ) là độc lập:

        \begin{equation*}
            f(X_1=x_1,X_2=x_2,\dots, X_n=x_n \vert \lambda) = \prod_{i=1}^n f(X_i = x_i \vert \lambda)
        \end{equation*}

        với $n$ là số quan sát trong tập dữ liệu.

        Vậy phân phối của dữ liệu là:

        \begin{equation*}
            \begin{aligned}
                f(X_1=x_1,X_2=x_2,\dots, X_n=x_n \vert \lambda) &= \prod_{i=1}^n f(X_i = x_i \vert \lambda) \\
                &= \prod_{i=1}^n e^{-\lambda} \dfrac{\lambda^{x_1}}{x_1!} \\
                &= e^{-n\lambda} \dfrac{\lambda^{x_1 + x_2 + \dots + x_n}}{x_1! x_2! \dots x_n!} \\
                &= e^{-n\lambda} \dfrac{\lambda^{\sum_{i=1}^n x_i}}{\prod_{i=1}^n x_i!}
            \end{aligned}
        \end{equation*}

        \item Xác định phân phối hậu nghiệm của tỷ lệ sinh khi phân phối tiên nghiệm là:

        Ta gọi $g(\lambda)$ là phân phối tiên nghiệm của $\lambda$.
        Phân phối hậu nghiệm được tính bởi công thức:

        \begin{equation*}
            g(\lambda \vert X_1 =x_1, X_2=x_2, \dots, X_n=x_n) = \dfrac{g(\lambda) f(X_1=x_1,X_2=x_2,\dots, X_n=x_n \vert \lambda)}{\int g(\lambda) f(X_1=x_1,X_2=x_2,\dots, X_n=x_n \vert \lambda) d \lambda}
        \end{equation*}

        \begin{enumerate}[label=(\alph*)]
            \item Phân phối đều
            
            Ta giả sử phân phối tiên nghiệm của $\lambda$ là:

            \begin{equation*}
                g(\lambda) = \begin{cases}
                    \dfrac{1}{a} \text{ nếu } \lambda \in \lbrack 0; a\rbrack (a > 0) \\
                    0 \text{ nếu ngược lại}
                \end{cases}
            \end{equation*}

            Như vậy ta có phân phối hậu nghiệm $g(\lambda \vert X_1 =x_1, X_2=x_2, \dots, X_n=x_n)$ của $\lambda$ là:

            \begin{equation*}
                \begin{aligned}
                    g(\lambda \vert X_1 =x_1, X_2=x_2, \dots, X_n=x_n) &= \dfrac{g(\lambda) f(X_1=x_1,X_2=x_2,\dots, X_n=x_n \vert \lambda)}{\int g(\lambda) f(X_1=x_1,X_2=x_2,\dots, X_n=x_n \vert \lambda) d \lambda} \\
                    &= \dfrac{\dfrac{1}{a} e^{-n\lambda} \dfrac{\lambda^{\sum_{i=1}^n x_i}}{\prod_{i=1}^n x_i!}}{\displaystyle\int_{0}^{a} \dfrac{1}{a} e^{-n\lambda} \dfrac{\lambda^{\sum_{i=1}^n x_i}}{\prod_{i=1}^n x_i!} d \lambda} \\
                    &= \dfrac{\dfrac{1}{a} e^{-n \lambda} \dfrac{(n\lambda)^{\sum_{i=1}^n x_i}}{\big(n^{\sum_{i=1}^n x_i}\big)\prod_{i=1}^n x_i!}}{\displaystyle\int_{0}^{a} \dfrac{1}{a} e^{-n\lambda} \dfrac{\lambda^{\sum_{i=1}^n x_i}}{\prod_{i=1}^n x_i!} d \lambda} \\
                    &= \dfrac{\dfrac{1}{a n^{(\sum_{i=1}^n x_i) + 1}\prod_{i=1}^n x_i !}  ne^{-n\lambda} (n\lambda)^{\sum_{i=1}^n x_i}}{\dfrac{1}{a \prod_{i=1}^n x_i!} \displaystyle \int_{0}^a e^{-n\lambda}\lambda^{\sum_{i=1}^n x_i}d\lambda} \\
                    &= \dfrac{n e^{-n\lambda}(n\lambda)^{\sum_{i=1}^n x_i}}{n^{(\sum_{i=1}^n x_i) + 1}\displaystyle\int_{0}^a e^{-n\lambda} \lambda^{\sum_{i=1}^n x_i}d\lambda}
                \end{aligned}
            \end{equation*}

            Ta nhận thấy các đại lượng $a$ (theo giả thiết về phân phối tiên nghiệm), $n^{\sum_{i=1}^n x_i}$  (tập dữ liệu ta đã biết), $\displaystyle\int_{0}^a e^{-n\lambda} \lambda^{\sum_{i=1}^n x_i}d\lambda$ đều là hằng số.
            Nên phân phối hậu nghiệm $g(\lambda \vert X_1 =x_1, X_2=x_2, \dots, X_n=x_n)$ của $\lambda$ có dạng:

            \begin{equation*}
                \begin{aligned}
                    g(\lambda \vert X_1 =x_1, X_2=x_2, \dots, X_n=x_n) &= \dfrac{n e^{-n\lambda}(n\lambda)^{\sum_{i=1}^n x_i}}{n^{(\sum_{i=1}^n x_i) + 1}\displaystyle\int_{0}^a e^{-n\lambda} \lambda^{\sum_{i=1}^n x_i}d\lambda} \\
                    &= \dfrac{n e^{-n\lambda}(n\lambda)^{\big\lbrack(\sum_{i=1}^n x_i) + 1\big\rbrack - 1}}{\Gamma\big((\sum_{i=1}^n x_i) + 1\big)}.\dfrac{\Gamma\big((\sum_{i=1}^n x_i) + 1\big)}{n^{(\sum_{i=1}^n x_i) + 1}\displaystyle\int_{0}^a e^{-n\lambda} \lambda^{\sum_{i=1}^n x_i}d\lambda} \\
                    &\propto \dfrac{n e^{-n\lambda}(n\lambda)^{\big\lbrack(\sum_{i=1}^n x_i) + 1\big\rbrack - 1}}{\Gamma\big((\sum_{i=1}^n x_i) + 1\big)}
                \end{aligned}
            \end{equation*}

            với hàm $\Gamma(\alpha)$ được định nghĩa bằng:

            \begin{equation*}
                \Gamma(\alpha) = \int_{0}^{\infty} e^{-y} y^{\alpha - 1} dy
            \end{equation*}

            Ta chú ý phân phối hậu nghiệm của $\lambda$ nên $\lambda$ là biến.
            Mặt khác, ta có hàm mật độ xác suất của phân phối gamma (ta đặt biến ngẫu nhiên là $\lambda$ thay vì $x$ để dễ thấy sự tương tự giữa phân phối hậu nghiệm của $\lambda$ và phân phối gamma):

            \begin{equation*}
                \text{gamma}(\lambda; n, \alpha) = \begin{cases}
                    \dfrac{n e^{-n\lambda} (n\lambda)^{\alpha - 1}}{\Gamma(\alpha)} \text{ nếu } \lambda > 0 \\
                    0 \text{ nếu ngược lại}
                \end{cases}
            \end{equation*}

            Vì vậy:

            \begin{equation*}
                g(\lambda \vert X_1 =x_1, X_2=x_2, \dots, X_n=x_n) \propto \text{gamma}\big(\lambda; n, (\sum_{i=1}^n x_i) + 1\big)
            \end{equation*}

            \item Phân phối gamma
            
            Ta giả sử phân phối tiên nghiệm của $\lambda$ là:

            \begin{equation*}
                g(\lambda) = \text{gamma}(\lambda; r, \alpha) = \begin{cases}
                    \dfrac{r e^{-r\lambda} (r\lambda)^{\alpha - 1}}{\Gamma(\alpha)}=\dfrac{r^{\alpha}e^{-r\lambda} \lambda^{\alpha - 1}}{\Gamma(\alpha)} \text{ nếu } \lambda > 0 \\
                    0 \text{ nếu ngược lại}
                \end{cases}
            \end{equation*}

            Như vậy ta có phân phối hậu nghiệm $g(\lambda \vert X_1 =x_1, X_2=x_2, \dots, X_n=x_n)$ của $\lambda$ là:

            \begin{equation*}
                \begin{aligned}
                    &g(\lambda \vert X_1 =x_1, X_2=x_2, \dots, X_n=x_n) \\
                    &= \dfrac{g(\lambda) f(X_1=x_1,X_2=x_2,\dots, X_n=x_n \vert \lambda)}{\int g(\lambda) f(X_1=x_1,X_2=x_2,\dots, X_n=x_n \vert \lambda) d \lambda} \\
                    &= \dfrac{\dfrac{r^{\alpha}e^{-r\lambda} \lambda^{\alpha - 1}}{\Gamma(\alpha)} \dfrac{e^{-n\lambda} \lambda^{\sum_{i=1}^n x_i}}{\prod_{i=1}^n x_i!}}{\displaystyle \int_{0}^{\infty} \dfrac{r^{\alpha}e^{-r\lambda} \lambda^{\alpha - 1}}{\Gamma(\alpha)}  \dfrac{e^{-n\lambda}\lambda^{\sum_{i=1}^n x_i}}{\prod_{i=1}^n x_i!} d \lambda} \\
                    &= \dfrac{r^{\alpha}e^{-(n+r)\lambda} \lambda^{(\sum_{i=1}^n x_i) + \alpha - 1}}{\Gamma(\alpha)\prod_{i=1}^n x_i! \displaystyle \int_{0}^{\infty} \dfrac{r^{\alpha}e^{-r\lambda} \lambda^{\alpha - 1}}{\Gamma(\alpha)}  \dfrac{e^{-n\lambda}\lambda^{\sum_{i=1}^n x_i}}{\prod_{i=1}^n x_i!} d \lambda} \\
                    &= \dfrac{r^{\alpha}e^{-(n+r)\lambda} \lambda^{\sum_{i=1}^n x_i + \alpha - 1}}{\Gamma(\alpha)\prod_{i=1}^n x_i! \dfrac{r^{\alpha}}{\Gamma(\alpha)\prod_{i=1}^n x_i!} \displaystyle \int_{0}^{\infty} e^{-(n+r)\lambda} \lambda^{\sum_{i=1}^n x_i + \alpha - 1} d \lambda} \\
                    &= \dfrac{e^{-(n+r)\lambda} \lambda^{(\sum_{i=1}^n x_i) + \alpha - 1}}{\displaystyle \int_{0}^{\infty} e^{-(n+r)\lambda} \lambda^{(\sum_{i=1}^n x_i) + \alpha - 1} d \lambda} \\
                    &= \dfrac{(n+r)e^{-(n+r)\lambda}\lambda^{(\sum_{i=1}^n x_i) + \alpha - 1}}{(n+r)^{(\sum_{i=1}^n x_i) + \alpha} .\dfrac{1}{(n+r)^{(\sum_{i=1}^n x_i) + \alpha}}\displaystyle\int_{0}^{\infty}e^{-(n+r)\lambda}\lambda^{(\sum_{i=1}^n x_i) + \alpha - 1}d\lbrack (n+r)\lambda \rbrack} \\
                    &= \dfrac{(n+r)e^{-(n+r)\lambda}\lambda^{(\sum_{i=1}^n x_i) + \alpha - 1}}{\displaystyle\int_{0}^{\infty}e^{-(n+r)\lambda}\lambda^{(\sum_{i=1}^n x_i) + \alpha - 1}d\lbrack (n+r)\lambda \rbrack} \text{ (mẫu số chính là } \Gamma\big( (\sum_{i=1}^n x_i) + \alpha \big)) \\
                    &= \dfrac{(n+r)e^{-(n+r)\lambda}\lambda^{(\sum_{i=1}^n x_i) + \alpha - 1}}{\Gamma\big( (\sum_{i=1}^n x_i) + \alpha \big)} \\
                    &= \text{gamma}\big(\lambda; (n+r), (\sum_{i=1}^n x_i) + \alpha \big)
                \end{aligned}
            \end{equation*}

            Như vậy ta nhận thấy phân phối hậu nghiệm $g(\lambda \vert X_1 =x_1, X_2=x_2, \dots, X_n=x_n)$ của $\lambda$ cũng chính là phân phối gamma khi phân phối tiên nghiệm của $\lambda$ là phân phối gamma.
        \end{enumerate}
        \item Đặc điểm của các phân phối hậu nghiệm tìm được là:
        
        Như ta đã chứng minh, khi phân phối tiên nghiệm là phân phối đều hoặc phân phối gamma thì phân phối hậu nghiệm đều có dạng phân phối gamma.

        
    \end{enumerate}
\end{loigiai}

\newpage
\printbibliography[title={TÀI LIỆU THAM KHẢO}]
\end{document}